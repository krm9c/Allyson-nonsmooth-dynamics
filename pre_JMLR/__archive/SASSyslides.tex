%----------------------------------------------------------------------------------------
%    PACKAGES AND THEMES
%----------------------------------------------------------------------------------------

\documentclass[aspectratio=169,xcolor=dvipsnames]{beamer}
%\usetheme{SimpleDarkBlue}
%\usetheme{AnnArbor}
%\usetheme{Antibes}
%\usetheme{Bergen}
%\usetheme{Berkeley}
%\usetheme{Berlin}
%\usetheme{Boadilla}
%\usetheme{boxes}
%\usetheme{CambridgeUS}
%\usetheme{Copenhagen}
%\usetheme{Darmstadt}
%\usetheme{default}
%\usetheme{Frankfurt}
%\usetheme{Goettingen}
%\usetheme{Hannover}
%\usetheme{Ilmenau}
%\usetheme{JuanLesPins}
%\usetheme{Luebeck}
\usetheme{Madrid}

%\usetheme{Malmoe}
%\usetheme{Marburg}
%\usetheme{Montpellier}
%\usetheme{PaloAlto}
%\usetheme{Pittsburgh}
%\usetheme{Rochester}
%\usetheme{Singapore}
%\usetheme{Szeged}
%\usetheme{Warsaw}

\usepackage{hyperref}
\usepackage{graphicx} % Allows including images
\usepackage{booktabs} % Allows the use of \toprule, \midrule and \bottomrule in tables
\usepackage{amssymb,amsthm,amsmath,tikz,pgf,mathtools,subfigure}
\usepackage[lined,linesnumbered,ruled]{algorithm2e}
\usepackage{epsfig,amsmath, amsfonts,amstext, amsthm, latexsym, graphicx}
\usepackage{amssymb,amsthm,amsmath,tikz,pgf,mathtools,subfigure}
\usepackage[lined,linesnumbered,ruled]{algorithm2e}
\usepackage{epsfig,amsmath, amsfonts,amstext, amsthm, latexsym, graphicx}
\usepackage{epstopdf}
\usepackage{epstopdf}
\usepackage{amsmath}
\usepackage{graphicx}
\usepackage{booktabs}
\usepackage{caption}
\usepackage[x11names]{xcolor}
\usepackage{tikz}
\usepackage{pifont}
\usepackage{color,soul}
\usepackage[dvipsnames]{xcolor}
\usepackage{comment}
\usepackage{wasysym}
%\usepackage{natbib}
%\usepackage{bibentry}
\usepackage[backend = biber]{biblatex}
%\bibliography{NewNNbib}
\addbibresource{NewNNbib.bib}
\AtBeginBibliography{\small}

\definecolor{UBCblue}{rgb}{0.04706, 0.13725, 0.26667} % UBC Blue (primary)
\definecolor{UBCblue}{rgb}{0.16078, 0.33725, 0.513725} % UBC Blue (primary)
\definecolor{UBCblue}{rgb}{.10588,.30196,.41176} % UBC Blue (primary)
\usecolortheme[named=UBCblue]{structure}

\setbeamertemplate{theorems}[numbered]
\newcommand{\abs}[1]{\lvert #1 \rvert}
\newcommand{\bigabs}[1]{\left \lvert #1 \right \rvert}
\newcommand{\fhat}{\hat{f}}
\def\MPlus{\ensuremath{\mathbin{\raisebox{-.1em}{\scalebox{.67}{\Plus}}}}} 



\newtheorem{proposition}[theorem]{Proposition}

\newtheorem{remark}[theorem]{Remark}
\usetikzlibrary{graphs}


\makeatletter
\renewcommand\@makefnmark{\hbox{\@textsuperscript{\normalfont[\@thefnmark]}}}
\renewcommand\@makefntext[1]{{\normalfont[\@thefnmark]}\enspace #1}

\renewrobustcmd{\blx@mkbibfootnote}[2]{%
  \iftoggle{blx@footnote}
    {\blx@warning{Nested notes}%
     \addspace\mkbibparens{#2}}
    {\unspace
     \ifnum\blx@notetype=\tw@
       \expandafter\@firstoftwo
     \else
       \expandafter\@secondoftwo
     \fi
       {\csuse{blx@theendnote#1}{\protecting{\blxmkbibnote{end}{#2}}}}
       {\csuse{footnote}[frame]{\protecting{\blxmkbibnote{foot}{#2}}}}}}
\makeatother

\AtBeginSection[]{
  \begin{frame}
  \vfill
  \centering
  \begin{beamercolorbox}[sep=8pt,center,shadow=true,rounded=true]{title}
    \usebeamerfont{title}\insertsectionhead\par%
  \end{beamercolorbox}
  \vfill
  \end{frame}
}


%----------------------------------------------------------------------------------------
%    TITLE PAGE
%----------------------------------------------------------------------------------------

\title[The Effect of Architecture on Learning Behavior]{The Effect of Architecture on the Learning Behavior of Deep Neural Networks}

\author[Hahn and Raghavan]{Allyson Hahn and Krishnan Raghavan}

\institute[NIU and ANL]
{    Northern Illinois University\\
    Argonne National Laboratory
   
     % Your institution for the title page
}
\date{July 9, 2025} % Date, can be changed to a custom date

%----------------------------------------------------------------------------------------
%    PRESENTATION SLIDES
%----------------------------------------------------------------------------------------

\begin{document}

\begin{frame}
    % Print the title page as the first slide
    \titlepage
\end{frame}


\begin{frame}{Motivation: Increasing Search Spaces in Continual Learning }
    \begin{center}
        \includegraphics[width =1\textwidth]{Figures/w1sassy.png} \\
    \end{center}
        
        \centering Figure: Neural Network weights search space for task 1.
        %\centering {\color{WildStrawberry} True, because the cost over a large dataset can be written as a sum of cost over $N$ individual batches.}
    \end{frame}

 \begin{frame}{Motivation: Increasing Search Spaces in Continual Learning }
    \begin{center}
        \includegraphics[width =0.9\textwidth]{Figures/w3sassypng.png} \\
    \end{center}
        
        \centering Figure: Neural Network weights search space for tasks 1 and 2.
        %\centering {\color{WildStrawberry} True, because the cost over a large dataset can be written as a sum of cost over $N$ individual batches.}
    \end{frame}


\begin{frame}{Method Overview}
    \begin{columns}[c] % The "c" option specifies centered vertical alignment while the "t" option is used for top vertical alignment

        \column{.33\textwidth} % Left column and width
        \begin{itemize}
            \item \textbf{Idea:} Rather than fixing the architecture at $\psi^*$ for all tasks, allowing an architecture search at each step affords us the opportunity for a larger intersection space.
            \item \textbf{Low Rank Transfer:} A method which allows us to change the size of the weights matrix according to the new architecture while transferring learning.
        \end{itemize}
        
        \column{.65\textwidth} % Right column and width
        \begin{figure}
            \centering
            \includegraphics[width=1.17\linewidth]{Figures/method.png}
        \end{figure}
    \end{columns}
\end{frame}


\begin{frame}{Transfer of Learning to New Architecture: Low Rank Transfer}
    
    \begin{columns}[c] % The "c" option specifies centered vertical alignment while the "t" option is used for top vertical alignment
        \column{.25\textwidth} % Left column and width
        \begin{itemize}
            \item \textbf{Question:} How do we transfer learning to a new architecture in practice.
            \item Here we consider \underline{only} the case where we change \textit{the number of neurons per layer}.
        \end{itemize}
                 
        
        \column{.7\textwidth} % Right column and width
        \begin{figure}
            \centering
            \includegraphics[width=.91\linewidth]{Figures/CLsolution1.png}
        \end{figure}
    \end{columns}
\end{frame}


\begin{frame}{Method Overview}
        \begin{figure}
            \centering
            \includegraphics[width=.9\linewidth]{Figures/method.png}
        \end{figure}
\end{frame}

\begin{frame}{Regression Experiment 1: Changing Architecture on a Single Task}
    \begin{figure}
        \centering
        \includegraphics[width=.95\linewidth]{Figures/regtask1.png}
        \label{fig:enter-label}
    \end{figure}
\end{frame}

\begin{frame}{Regression Experiment 2: Changing Architecture on Every Task}
    \begin{figure}
        \centering
        \includegraphics[width=1\linewidth]{Figures/regtask2.png}
    \end{figure}
\end{frame}


\begin{frame}{Classification Experiment: Changing Architecture on Every Task}
    \begin{figure}
        \centering
        \includegraphics[width=1\linewidth]{Figures/CNN2.png}
    \end{figure}
\end{frame}


\end{document}